\documentclass[12pt]{article}
 
\usepackage[margin=1in]{geometry}
\usepackage{amsmath,amsthm,amssymb}
\usepackage{mathtools}
\DeclarePairedDelimiter{\ceil}{\lceil}{\rceil}
%\usepackage{mathptmx}
\usepackage{accents}
\usepackage{comment}
\usepackage{graphicx}
\usepackage{IEEEtrantools}
 \usepackage{float}
 \usepackage{mathrsfs}
 
\newcommand{\N}{\mathbb{N}}
\newcommand{\Z}{\mathbb{Z}}
\newcommand{\R}{\mathbb{R}}
\newcommand{\Q}{\mathbb{Q}}
\newcommand*\conj[1]{\bar{#1}}
\newcommand*\mean[1]{\bar{#1}}
\newcommand\widebar[1]{\mathop{\overline{#1}}}


\newcommand{\cc}{{\mathbb C}}
\newcommand{\rr}{{\mathbb R}}
\newcommand{\qq}{{\mathbb Q}}
\newcommand{\nn}{\mathbb N}
\newcommand{\zz}{\mathbb Z}
\newcommand{\aaa}{{\mathcal A}}
\newcommand{\bbb}{{\mathcal B}}
\newcommand{\rrr}{{\mathcal R}}
\newcommand{\fff}{{\mathcal F}}
\newcommand{\ppp}{{\mathcal P}}
\newcommand{\eps}{\varepsilon}
\newcommand{\vv}{{\mathbf v}}
\newcommand{\ww}{{\mathbf w}}
\newcommand{\xx}{{\mathbf x}}
\newcommand{\ds}{\displaystyle}
\newcommand{\Om}{\Omega}
\newcommand{\dd}{\mathop{}\,\mathrm{d}}
\newcommand{\ud}{\, \mathrm{d}}
\newcommand{\seq}[1]{\left\{#1\right\}_{n=1}^\infty}
\newcommand{\isp}[1]{\quad\text{#1}\quad}
\newcommand*\diff{\mathop{}\!\mathrm{d}}

\DeclareMathOperator{\imag}{Im}
\DeclareMathOperator{\re}{Re}
\DeclareMathOperator{\diam}{diam}
\DeclareMathOperator{\Tr}{Tr}
\DeclareMathOperator{\cis}{cis}

\def\upint{\mathchoice%
    {\mkern13mu\overline{\vphantom{\intop}\mkern7mu}\mkern-20mu}%
    {\mkern7mu\overline{\vphantom{\intop}\mkern7mu}\mkern-14mu}%
    {\mkern7mu\overline{\vphantom{\intop}\mkern7mu}\mkern-14mu}%
    {\mkern7mu\overline{\vphantom{\intop}\mkern7mu}\mkern-14mu}%
  \int}
\def\lowint{\mkern3mu\underline{\vphantom{\intop}\mkern7mu}\mkern-10mu\int}




\newenvironment{theorem}[2][Theorem]{\begin{trivlist}
\item[\hskip \labelsep {\bfseries #1}\hskip \labelsep {\bfseries #2.}]}{\end{trivlist}}
\newenvironment{lemma}[2][Lemma]{\begin{trivlist}
\item[\hskip \labelsep {\bfseries #1}\hskip \labelsep {\bfseries #2.}]}{\end{trivlist}}
\newenvironment{exercise}[2][Exercise]{\begin{trivlist}
\item[\hskip \labelsep {\bfseries #1}\hskip \labelsep {\bfseries #2.}]}{\end{trivlist}}
\newenvironment{problem}[2][Problem]{\begin{trivlist}
\item[\hskip \labelsep {\bfseries #1}\hskip \labelsep {\bfseries #2.}]}{\end{trivlist}}
\newenvironment{question}[2][Question]{\begin{trivlist}
\item[\hskip \labelsep {\bfseries #1}\hskip \labelsep {\bfseries #2.}]}{\end{trivlist}}
\newenvironment{corollary}[2][Corollary]{\begin{trivlist}
\item[\hskip \labelsep {\bfseries #1}\hskip \labelsep {\bfseries #2.}]}{\end{trivlist}}

\newenvironment{solution}{\begin{proof}[Solution]}{\end{proof}}
 
\begin{document}
 
% --------------------------------------------------------------
%                         Start here
% --------------------------------------------------------------
\title{Proof}
\author{Ethan}
\date{\today}
\maketitle
\hbadness=99999
\hfuzz=50pt
\noindent
Let $\Omega \subseteq \mathbb{C}^n$ be an open bounded connected set, and let $L^2(\Omega,\lambda)$ denote the space of square integrable functions (with weight $\lambda$). Let $A^2(\Omega, \lambda)$ denote the space of holomorphic functions in $L^2(\Omega,\lambda)$. Since $A^2(\Omega,\lambda)$ is closed in $L^2(\Omega,\lambda)$, there exists an orthogonal projection operator $P: L^2(\Omega,\lambda) \rightarrow A^2(\Omega,\lambda)$. We claim that $\Vert P \Vert = 1$. First, we recall that
\[
\Vert P \Vert = \sup\{\Vert Pf \Vert: f \in L^2(\Omega, \lambda), \Vert f \Vert \leq 1\}
\] We first show that $\Vert P \Vert \leq 1$. Let $f \in L^2(\Omega, \lambda)$. Since $P$ is orthogonal, we find that
\begin{align}
&\Vert f \Vert^2 = \Vert Pf + (I-P)f \Vert^2 = \langle Pf + (I-P)f, Pf + (I - P)f \rangle =\\
& \langle Pf, Pf \rangle + \langle Pf, (I-P)f \rangle + \langle (I-P)f, Pf \rangle + \langle (I-P)f, (I-P)f \rangle = \\
& \langle Pf, Pf \rangle + \langle (I-P)f, (I-P)f \rangle = \Vert Pf \Vert^2 + \Vert(I-P)\Vert^2
\end{align} so that
\begin{align}
\Vert Pf \Vert^2 = \Vert f \Vert^2 - \Vert (I-P)f \Vert^2 \leq \Vert f \Vert^2
\end{align} and
\begin{align}
\Vert Pf \Vert \leq \Vert f \Vert \leq 1
\end{align} From this, we deduce that $\Vert P \Vert \leq 1$. Next, we want to show that $\Vert P \Vert \geq 1$. Let $f \in A^2(\Omega,\lambda)$ be such that $f \not \equiv 0$. Then $\Vert f \Vert > 0$ so that $f/\Vert f \Vert \in A^2(\Omega,\lambda)$. Then, we find that
\[
\Vert P(f/\Vert f \Vert) \Vert = \Vert f/\Vert f\Vert \Vert = 1
\] so that $\Vert P \Vert \geq 1$. My only concern is if $A^2(\Omega,\lambda) = \{0\}$.
\end{document} 