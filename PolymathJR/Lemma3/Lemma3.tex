\documentclass[12pt]{article}
 
\usepackage[margin=1in]{geometry}
\usepackage{amsmath,amsthm,amssymb}
\usepackage{mathtools}
\DeclarePairedDelimiter{\ceil}{\lceil}{\rceil}
%\usepackage{mathptmx}
\usepackage{accents}
\usepackage{comment}
\usepackage{graphicx}
\usepackage{IEEEtrantools}
 \usepackage{float}
 \usepackage{mathrsfs}
 
\newcommand{\N}{\mathbb{N}}
\newcommand{\Z}{\mathbb{Z}}
\newcommand{\R}{\mathbb{R}}
\newcommand{\Q}{\mathbb{Q}}
\newcommand*\conj[1]{\bar{#1}}
\newcommand*\mean[1]{\bar{#1}}
\newcommand\widebar[1]{\mathop{\overline{#1}}}


\newcommand{\cc}{{\mathbb C}}
\newcommand{\rr}{{\mathbb R}}
\newcommand{\qq}{{\mathbb Q}}
\newcommand{\nn}{\mathbb N}
\newcommand{\zz}{\mathbb Z}
\newcommand{\aaa}{{\mathcal A}}
\newcommand{\bbb}{{\mathcal B}}
\newcommand{\rrr}{{\mathcal R}}
\newcommand{\fff}{{\mathcal F}}
\newcommand{\ppp}{{\mathcal P}}
\newcommand{\eps}{\varepsilon}
\newcommand{\vv}{{\mathbf v}}
\newcommand{\ww}{{\mathbf w}}
\newcommand{\xx}{{\mathbf x}}
\newcommand{\ds}{\displaystyle}
\newcommand{\Om}{\Omega}
\newcommand{\dd}{\mathop{}\,\mathrm{d}}
\newcommand{\ud}{\, \mathrm{d}}
\newcommand{\seq}[1]{\left\{#1\right\}_{n=1}^\infty}
\newcommand{\isp}[1]{\quad\text{#1}\quad}
\newcommand*\diff{\mathop{}\!\mathrm{d}}

\DeclareMathOperator{\imag}{Im}
\DeclareMathOperator{\re}{Re}
\DeclareMathOperator{\diam}{diam}
\DeclareMathOperator{\Tr}{Tr}
\DeclareMathOperator{\cis}{cis}

\def\upint{\mathchoice%
    {\mkern13mu\overline{\vphantom{\intop}\mkern7mu}\mkern-20mu}%
    {\mkern7mu\overline{\vphantom{\intop}\mkern7mu}\mkern-14mu}%
    {\mkern7mu\overline{\vphantom{\intop}\mkern7mu}\mkern-14mu}%
    {\mkern7mu\overline{\vphantom{\intop}\mkern7mu}\mkern-14mu}%
  \int}
\def\lowint{\mkern3mu\underline{\vphantom{\intop}\mkern7mu}\mkern-10mu\int}




\newenvironment{theorem}[2][Theorem]{\begin{trivlist}
\item[\hskip \labelsep {\bfseries #1}\hskip \labelsep {\bfseries #2.}]}{\end{trivlist}}
\newenvironment{lemma}[2][Lemma]{\begin{trivlist}
\item[\hskip \labelsep {\bfseries #1}\hskip \labelsep {\bfseries #2.}]}{\end{trivlist}}
\newenvironment{exercise}[2][Exercise]{\begin{trivlist}
\item[\hskip \labelsep {\bfseries #1}\hskip \labelsep {\bfseries #2.}]}{\end{trivlist}}
\newenvironment{problem}[2][Problem]{\begin{trivlist}
\item[\hskip \labelsep {\bfseries #1}\hskip \labelsep {\bfseries #2.}]}{\end{trivlist}}
\newenvironment{question}[2][Question]{\begin{trivlist}
\item[\hskip \labelsep {\bfseries #1}\hskip \labelsep {\bfseries #2.}]}{\end{trivlist}}
\newenvironment{corollary}[2][Corollary]{\begin{trivlist}
\item[\hskip \labelsep {\bfseries #1}\hskip \labelsep {\bfseries #2.}]}{\end{trivlist}}

\newenvironment{solution}{\begin{proof}[Solution]}{\end{proof}}
 
\begin{document}
 
% --------------------------------------------------------------
%                         Start here
% --------------------------------------------------------------
\title{Computations}
\author{Ethan Martirosyan}
\date{\today}
\maketitle
\hbadness=99999
\hfuzz=50pt
\section*{Lemma 3.1}
Also, I am trying to prove Lemma $3.1$ from Axler and Zheng's paper, which says that $$\tilde{S} \circ \varphi_z = \widetilde{S_z}$$ First, I note that $$\tilde{S} \circ \varphi_z (w) = \tilde{S}(\varphi_z(w)) = \langle Sk_{\varphi_z(w)}, k_{\varphi_z(w)} \rangle$$ and that $$\widetilde{S_z}(w) = \langle S_z k_w, k_w \rangle = \langle U_z S U_z k_w, k_w \rangle = \langle S U_z k_w,  U_z k_w \rangle$$ Thus it suffices to prove that $$U_z k_w = k_{\varphi_z(w)}$$ By the definition of $U_z$, I have $$U_z k_w = (k_w \circ \varphi_z) \varphi_z^\prime$$ so I am trying to prove that $$(k_w \circ \varphi_z) \varphi_z^\prime = k_{\varphi_z(w)}$$ or $$(k_w \circ \varphi_z(v)) \varphi_z^\prime(v) = k_{\varphi_z(w)}(v)$$ To show this, I appeal to the formula from office hours: $$K_U(z, \overline{\zeta}) = \det Df(z) \overline{\det Df(\zeta)}  K_V(f(z), \overline{f(\zeta)})$$ Here I take $U = V = \mathbb{D}$, $f = \varphi_z$, $z = w$ and $\zeta = \varphi_z(v)$. Thus the above formula becomes $$K_\mathbb{D}(w,\overline{\varphi_z(v)}) = \det D \varphi_z(w) \overline{\det D \varphi_z(\varphi_z(v))} K_\mathbb{D}(\varphi_z(w), \overline{v})$$ Then, I compute $$K_\mathbb{D}(w,\overline{\varphi_z(v)}) = \overline{K_w(\varphi_z(v))} = \overline{\frac{k_w(\varphi_z(v))}{1 - \vert w \vert^2}}$$ $$K_\mathbb{D}(\varphi_z(w), \overline{v}) = \overline{K_{\varphi_z(w)}(v)} = \overline{\frac{k_{\varphi_z(w)}(v)}{1- \vert \varphi_z(w) \vert^2}}$$ $$\det D \varphi_z(w) = \varphi_z^\prime(w) = \frac{\vert z \vert^2 - 1}{(1-\overline{z}w)^2}$$ and finally $$\overline{\det D \varphi_z(\varphi_z(v))} = \overline{\varphi_z^\prime(\varphi_z(v))} = \overline{\frac{1}{\varphi_z^\prime(v)}}$$ Putting it all together, I obtain $$\overline{\frac{k_w(\varphi_z(v))}{1 - \vert w \vert^2}} =  \frac{\vert z \vert^2 - 1}{(1-\overline{z}w)^2} \cdot  \overline{\frac{1}{\varphi_z^\prime(v)}} \cdot  \overline{\frac{k_{\varphi_z(w)}(v)}{1- \vert \varphi_z(w) \vert^2}}$$ I take the complex conjugate to obtain $$\frac{k_w(\varphi_z(v))}{1 - \vert w \vert^2} =  \frac{\vert z \vert^2 - 1}{(1-z\overline{w})^2} \cdot  \frac{1}{\varphi_z^\prime(v)} \cdot  \frac{k_{\varphi_z(w)}(v)}{1- \vert \varphi_z(w) \vert^2}$$ Using the fact that $$1 - \vert \varphi_z(w)\vert^2 = \frac{(1-\vert z \vert^2)(1- \vert w \vert^2)}{\vert 1 - z \overline{w} \vert^2}$$ the above equation becomes $$\frac{k_w(\varphi_z(v))}{1 - \vert w \vert^2} =  \frac{\vert z \vert^2 - 1}{(1-z\overline{w})^2} \cdot  \frac{1}{\varphi_z^\prime(v)} \cdot  \frac{k_{\varphi_z(w)}(v)}{\frac{(1-\vert z \vert^2)(1- \vert w \vert^2)}{\vert 1 - z \overline{w} \vert^2}}$$ Simplification yields $$k_w(\varphi_z(v)) \varphi_z^\prime(v) = k_{\varphi_z(w)}(v) \cdot -\frac{\vert 1- z \overline{w} \vert^2}{(1 - z \overline{w})^2} $$ I can't seem to get rid of the $$-\frac{\vert 1- z \overline{w} \vert^2}{(1 - z \overline{w})^2} $$ term. I would be grateful if anyone could tell me where my calculations went wrong.
\end{document} 