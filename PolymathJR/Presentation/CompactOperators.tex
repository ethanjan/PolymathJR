\documentclass{beamer}
\usetheme{Madrid}
%Information to be included in the title page:
\title{Toeplitz Operators and Bergman Spaces}
\author{}
\institute{Polymath Jr}
\date{2024}


\begin{document}

\frame{\titlepage}


\begin{frame}

\begin{itemize}
\item $L^2(\mathbb{D}, dA)$: space of square-integrable functions on $\mathbb{D}$
\item $L_a^2$: closed subspace of analytic functions in $L^2(\mathbb{D}, dA)$
\item $P: L^2(\mathbb{D},dA) \rightarrow L_a^2$: orthogonal projection operator
\item $T_u: L_a^2 \rightarrow L_a^2$: Toeplitz operator with symbol $u$. $T_u(f) = P(u f)$
\item $K_z \in L_a^2$: Bergman reproducing kernel. $f(z) = \langle f, K_z \rangle$ 
\item $k_z \in L_a^2$: normalized Bergman reproducing kernel. $k_z = K_z/\Vert K_z \Vert_2$
\item $\tilde{S}: \mathbb{D} \rightarrow \mathbb{C}$: Berezin transform of $S$. $\tilde{S}(z) = \langle Sk_z, k_z \rangle$.
\item $\varphi_z: \mathbb{D} \rightarrow \mathbb{D}$: automorphism of unit disk. $\varphi_z(w) = (z - w)/(1-\overline{z}w)$
\item $U_z: L_a^2 \rightarrow L_a^2$: $U_z f = (f\circ \varphi_z)\varphi_z^\prime$
\item $S_z: L_a^2 \rightarrow L_a^2$: $S_z = U_zSU_z$
\item $H_u: L_a^2 \rightarrow (L_a^2)^\perp$: Hankel operator with symbol $u$. $H_u(f) = (I-P)(uf)$
\end{itemize}
\end{frame}

\begin{frame}
\begin{block}{Theorem by Axler, Zheng$^{[1]}$}
Suppose $S$ is a finite sum of finite products of Toeplitz operators. Then the following are equivalent:
\begin{enumerate}[(i)]
            \item $S$ is compact
            \item $\Vert Sk_z \Vert_2 \rightarrow 0$ as $z \rightarrow \partial{\mathbb{D}}$
            \item $\tilde{S}(z) \rightarrow 0$ as $z \rightarrow \partial{\mathbb{D}}$
            \item $S_z 1 \rightarrow 0$ weakly in $L_a^2$ as $z \rightarrow \partial{\mathbb{D}}$
            \item $\Vert S_z 1 \Vert_2 \rightarrow 0$ as $z \rightarrow \partial{\mathbb{D}}$
            \item $\Vert S_z 1\Vert_p \rightarrow 0$ as $z \rightarrow \partial{\mathbb{D}}$ for all $p \in (1,\infty)$
        \end{enumerate}
        
\end{block}
	\tiny{Axler, Sheldon, and Dechao Zheng. “Compact Operators via the Berezin Transform.” Indiana University Mathematics Journal, vol. 47, no. 2, 1998, pp. 387–400. JSTOR, http://www.jstor.org/stable/24899675. Accessed 30 July 2024.}
\end{frame}

\begin{frame}
We define $U_z: A^2(\mathbb{D}) \rightarrow A^2(\mathbb{D})$ as follows: \(U_zf = (f \circ \varphi_z)\varphi_z^\prime\)
\begin{block}{Fact}
$U_z$ is unitary: $\langle U_z f, U_z f \rangle = \langle f, f \rangle$ for all $f \in A^2(\mathbb{D})$.
\end{block}
\begin{proof}
\[
\langle U_z f, U_z f \rangle = \int_\mathbb{D} \vert (U_zf)(w) \vert^2 dA(w) = \int_\mathbb{D} \vert (f \circ \varphi_z)(w) \vert^2 \vert \varphi_z^\prime(w)\vert^2 dA(w)
\] Let $\lambda = \varphi_z(w)$. Then $dA(\lambda) = \vert \varphi_z^\prime(w) \vert^2 dA(w)$ so that
\[
\int_\mathbb{D} \vert (f \circ \varphi_z)(w) \vert^2 \vert \varphi_z^\prime(w)\vert^2 dA(w) = \int_\mathbb{D} \vert f(\lambda) \vert^2 dA(\lambda) = \langle f, f \rangle
\]
\end{proof}
\end{frame}

\begin{frame}
\begin{block}{Fact}
$H_{\overline{f}}^* H_g = T_{fg} - T_f T_g$ for any symbols $f,g \in L^\infty(\mathbb{D},dA)$. 
\end{block}
\begin{proof}
First, we compute what $H_{\overline{f}}^*$ is. Let $a \in (L_a^2)^\perp$ and $b \in L_a^2$. Then,
\[
\langle H_{\overline{f}}^*a, b \rangle = \langle a, H_{\overline{f}} b \rangle = \langle a, (I-P)(\overline{f}b) \rangle = \langle a, \overline{f}b - P(\overline{f}b) \rangle 
\]
\[
= \langle a, \overline{f}b \rangle - \langle a,  P(\overline{f}b)\rangle = \langle fa, b \rangle - \langle Pa, \overline{f}b \rangle = \langle fa, b \rangle
\] Thus we deduce that $H_{\overline{f}}^*a = fa$. Now, we let $b \in L_a^2$. We obtain
\[
H_{\overline{f}}^* H_g(b) = P(H_{\overline{f}}^* H_g(b)) = P(H_{\overline{f}}^*(gb - P(gb)))  
\]
\[
= P(fgb - fP(gb)) = P(fgb) - P(fP(gb)) = T_{fg}(b) - T_f T_g(b)
\] thus proving that 
\[
H_{\overline{f}}^* H_g = T_{fg} - T_f T_g
\]
\end{proof}
\end{frame}

\begin{frame}
\begin{alertblock}{Lemma}
Let $S: L_a^2 \rightarrow L_a^2$ be a bounded operator. Then, $\tilde{S} \circ \varphi_z = \widetilde{S_z} $
\end{alertblock}
\begin{proof}
First, 
\[
\tilde{S} \circ \varphi_z (w) = \tilde{S}(\varphi_z(w)) = \langle Sk_{\varphi_z(w)}, k_{\varphi_z(w)} \rangle
\] and
\[
\widetilde{S_z}(w) = \langle S_z k_w, k_w \rangle = \langle U_z S U_z k_w, k_w \rangle = \langle S U_z k_w,  U_z k_w \rangle
\] Note that
\[
U_z k_w = (k_w \circ \varphi_z) \varphi_z^\prime
\] I appeal to the following formula$^{[2]}$: 
\[
K_U(z, \overline{\zeta}) = \det Df(z) \overline{\det Df(\zeta)}  K_V(f(z), \overline{f(\zeta)})
\] 
\end{proof}
\tiny{Lebl, Jiří. Tasty Bits of Several Complex Variables. Oklahoma State University. 3 Jun 2024.}
\end{frame}
\begin{frame}
\begin{proof}
Take $U = V = \mathbb{D}$, $f = \varphi_z$, $z = w$ and $\zeta = \varphi_z(v)$. Then
 \[
 K_\mathbb{D}(w,\overline{\varphi_z(v)}) = \det D \varphi_z(w) \overline{\det D \varphi_z(\varphi_z(v))} K_\mathbb{D}(\varphi_z(w), \overline{v})
 \] I compute 
 \[
 K_\mathbb{D}(w,\overline{\varphi_z(v)}) = \overline{K_w(\varphi_z(v))} = \overline{\frac{k_w(\varphi_z(v))}{1 - \vert w \vert^2}}
 \] 
 \[
 K_\mathbb{D}(\varphi_z(w), \overline{v}) = \overline{K_{\varphi_z(w)}(v)} = \overline{\frac{k_{\varphi_z(w)}(v)}{1- \vert \varphi_z(w) \vert^2}}
 \] 
 \[
 \det D \varphi_z(w) = \varphi_z^\prime(w) = \frac{\vert z \vert^2 - 1}{(1-\overline{z}w)^2}
 \] 
 \[
 \overline{\det D \varphi_z(\varphi_z(v))} = \overline{\varphi_z^\prime(\varphi_z(v))} = \overline{\frac{1}{\varphi_z^\prime(v)}}
 \]
 \end{proof}
\end{frame}

\begin{frame}
\begin{proof}
Putting it all together yields 
 \[
 \overline{\frac{k_w(\varphi_z(v))}{1 - \vert w \vert^2}} =  \frac{\vert z \vert^2 - 1}{(1-\overline{z}w)^2} \cdot  \overline{\frac{1}{\varphi_z^\prime(v)}} \cdot  \overline{\frac{k_{\varphi_z(w)}(v)}{1- \vert \varphi_z(w) \vert^2}}
 \] which simplifies to
 \[
 k_w(\varphi_z(v)) \varphi_z^\prime(v) = k_{\varphi_z(w)}(v) \cdot -\frac{\vert 1- z \overline{w} \vert^2}{(1 - z \overline{w})^2}
 \] With this formula, it can be shown that
 \[
 \widetilde{S_z}(w) = \langle SU_z k_w,  U_z k_w\rangle = \langle Sk_{\varphi_z(w)}, k_{\varphi_z(w)} \rangle = \tilde{S}(\varphi_z(w))
 \]
 \end{proof}
\end{frame}

\begin{frame}
\begin{block}{Fact}
$\phi$ vanishes on $\partial{\mathbb{D}} \implies T_\phi$ is compact.
\end{block}
Is this true on exponentially weighted spaces? Let $\lambda(z) = \exp(-1/(1-\vert z\vert^2))$ and $\phi(z) = 1 - \vert z \vert^2$. To show that $T_\phi: L_a^2(\mathbb{D},\lambda) \rightarrow L_a^2(\mathbb{D},\lambda)$ is not compact, it suffices to find a bounded sequence $\{f_n\}$ such that $\{T_\phi(f_n)\}$ does not contain a convergent subsequence.

Let $f_n(z)  = 1/(1-z^n)$. Then 
\[
\Vert f_1 \Vert \approx 0.468 \quad \Vert f_2 \Vert \approx 0.412 \quad \Vert f_3 \Vert \approx 0.396 
\]
\[
\Vert f_4 \Vert \approx  0.391 \quad \Vert f_5 \Vert \approx 0.388
\]
\end{frame} 
\begin{frame}
Thus $\{f_n\}$ is bounded. Furthermore,
\[
\Vert f_1 \phi \Vert \approx 0.331 \quad \Vert f_2 \phi \Vert \approx 0.302 \quad \Vert f_3 \phi \Vert \approx 0.296 
\]
\[
\Vert f_4 \phi \Vert \approx 0.294 \quad \Vert f_5 \phi \Vert \approx 0.294
\]
Note that
\[
\Vert T_\phi f_n \Vert = \Vert P(\phi f_n) \Vert \leq \Vert \phi f_n \Vert \leq \Vert f_n \Vert
\] so $\{T_\phi(f_n)\}$ is also bounded.
\end{frame}
\end{document}