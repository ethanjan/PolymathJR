\documentclass[12pt]{article}
 
\usepackage[margin=1in]{geometry}
\usepackage{amsmath,amsthm,amssymb}
\usepackage{mathtools}
\DeclarePairedDelimiter{\ceil}{\lceil}{\rceil}
%\usepackage{mathptmx}
\usepackage{accents}
\usepackage{comment}
\usepackage{graphicx}
\usepackage{IEEEtrantools}
 \usepackage{float}
 \usepackage{mathrsfs}
 
\newcommand{\N}{\mathbb{N}}
\newcommand{\Z}{\mathbb{Z}}
\newcommand{\R}{\mathbb{R}}
\newcommand{\Q}{\mathbb{Q}}
\newcommand*\conj[1]{\bar{#1}}
\newcommand*\mean[1]{\bar{#1}}
\newcommand\widebar[1]{\mathop{\overline{#1}}}


\newcommand{\cc}{{\mathbb C}}
\newcommand{\rr}{{\mathbb R}}
\newcommand{\qq}{{\mathbb Q}}
\newcommand{\nn}{\mathbb N}
\newcommand{\zz}{\mathbb Z}
\newcommand{\aaa}{{\mathcal A}}
\newcommand{\bbb}{{\mathcal B}}
\newcommand{\rrr}{{\mathcal R}}
\newcommand{\fff}{{\mathcal F}}
\newcommand{\ppp}{{\mathcal P}}
\newcommand{\eps}{\varepsilon}
\newcommand{\vv}{{\mathbf v}}
\newcommand{\ww}{{\mathbf w}}
\newcommand{\xx}{{\mathbf x}}
\newcommand{\ds}{\displaystyle}
\newcommand{\Om}{\Omega}
\newcommand{\dd}{\mathop{}\,\mathrm{d}}
\newcommand{\ud}{\, \mathrm{d}}
\newcommand{\seq}[1]{\left\{#1\right\}_{n=1}^\infty}
\newcommand{\isp}[1]{\quad\text{#1}\quad}
\newcommand*\diff{\mathop{}\!\mathrm{d}}

\DeclareMathOperator{\imag}{Im}
\DeclareMathOperator{\re}{Re}
\DeclareMathOperator{\diam}{diam}
\DeclareMathOperator{\Tr}{Tr}
\DeclareMathOperator{\cis}{cis}

\def\upint{\mathchoice%
    {\mkern13mu\overline{\vphantom{\intop}\mkern7mu}\mkern-20mu}%
    {\mkern7mu\overline{\vphantom{\intop}\mkern7mu}\mkern-14mu}%
    {\mkern7mu\overline{\vphantom{\intop}\mkern7mu}\mkern-14mu}%
    {\mkern7mu\overline{\vphantom{\intop}\mkern7mu}\mkern-14mu}%
  \int}
\def\lowint{\mkern3mu\underline{\vphantom{\intop}\mkern7mu}\mkern-10mu\int}



\newtheorem{theorem}{Theorem}

 \allowdisplaybreaks
\begin{document}
 
% --------------------------------------------------------------
%                         Start here
% --------------------------------------------------------------
\title{Report}
\author{Ethan Martirosyan}
\date{\today}
\maketitle
\hbadness=99999
\hfuzz=50pt
\section*{Previous Research}
During the first week, the mentors introduced the projects to us, and we were asked to rank the projects according to our preferences. I chose the projects regarding asymmetric colorings of graphs, ribbon knots, and Toeplitz operators on Bergman spaces, and I began reading the corresponding materials for each of these projects. At the end of the week, I was assigned to the Toeplitz operators on Bergman spaces group.
\par I spent the second week reading Axler and Zheng's paper. To discuss their main result, I must first introduce some notation. The Bergman space of the unit disk (which we denote $L_a^2$) is defined to be the set of all analytic functions in $L^2(\mathbb{D},dA)$ (which is the set of all square-integrable functions on the unit disk). It can be shown that $L_a^2$ is a closed subspace of the Hilbert space $L^2(\mathbb{D},dA)$. As such, there exists an orthogonal projection operator $P: L^2(\mathbb{D},dA) \rightarrow L_a^2$. Now, we define the Toeplitz operator $T_\phi: L_a^2 \rightarrow L_a^2$ with symbol $\phi \in L^\infty(\mathbb{D},dA)$ as follows: $T_\phi(f) = P(\phi f)$. Next, we define the Bergman reproducing kernel: for any $z \in \mathbb{D}$, the corresponding Bergman reproducing kernel is the function $K_z \in L_a^2$ such that $f(z) = \langle f, K_z \rangle$ for all $f \in L_a^2$. The normalized Bergman reproducing kernel is the function $k_z := K_z/\Vert K_z \Vert_2$. Finally, for any bounded operator $S$ on $L_a^2$, we define its Berezin transform $\tilde{S}$ as follows: $\tilde{S}(z) = \langle Sk_z, k_z \rangle$. Now that we have the definitions, we may state the main result of Axler and Zheng's paper:
\begin{theorem}
Suppose $S$ is a finite sum of finite products of Toeplitz operators. Then the following are equivalent:
\begin{enumerate}
\item $S$ is compact
\item $\Vert Sk_z \Vert_2 \rightarrow 0$ as $z \rightarrow \partial{\mathbb{D}}$
\item $\tilde{S}(z) \rightarrow 0$ as $z \rightarrow \partial{\mathbb{D}}$
\item $S_z 1 \rightarrow 0$ weakly in $L_a^2$ as $z \rightarrow \partial{\mathbb{D}}$
\item $\Vert S_z 1 \Vert_2 \rightarrow 0$ as $z \rightarrow \partial{\mathbb{D}}$
\item $\Vert S_z 1\Vert_p \rightarrow 0$ as $z \rightarrow \partial{\mathbb{D}}$ for all $p \in (1,\infty)$
\end{enumerate}
\end{theorem}
 The most important equivalence is that $S$ is compact if and only if its Berezin transform vanishes on the boundary. I will not restate the proof that they provide in the paper. However, I will go over some of the results in the paper and prove them. First they define the function $\varphi_z: \mathbb{D} \rightarrow \mathbb{D}$ as follows:
\begin{align*}
\varphi_z(w) = \frac{z-w}{1-\overline{z}w}
\end{align*} Then they claim that $\varphi_z \circ \varphi_z$ is the identity. To show this, we compute
\begin{align*}
&\varphi_z(\varphi_z(w)) = \frac{z-\varphi_z(w)}{1-\overline{z}\varphi_z(w)} = \frac{z-\frac{z-w}{1-\overline{z}w}}{1-\overline{z}\frac{z-w}{1-\overline{z}w}} = \frac{z(1-\overline{z}w) - (z-w)}{1-\overline{z}w - \overline{z}(z-w)} = \frac{z - \vert z \vert^2 w - z + w}{1 - \overline{z}w - \vert z \vert^2+ \overline{z}w}  \\ 
& = \frac{w - \vert z \vert^2 w}{1- \vert z \vert^2} = \frac{w(1- \vert z\vert^2)}{1- \vert z \vert^2} = w
\end{align*}
Next, they define the operator $U_z: L_a^2 \rightarrow L_a^2$ by $U_z f = (f \circ \varphi_z) \varphi_z^\prime$. They claim that this operator is unitary. We proceed to verify this:
\begin{align*}
&\langle U_z f, U_zf \rangle = \int_\mathbb{D} (U_z f)(w) \cdot \overline{(U_z f)(w)} dA(w) = \int_\mathbb{D} f(\varphi_z(w))\varphi_z^\prime(w) \overline{f(\varphi_z(w))\varphi_z^\prime(w)} dA(w)\\
&= \int_\mathbb{D} \vert f(\varphi_z(w)) \vert^2 \vert \varphi_z^\prime(w) \vert^2 dA(w)
\end{align*}
 We make the substitution $\lambda = \varphi_z(w)$. The domain remains the same because $\varphi_z$ is an automorphism of the unit disk $\mathbb{D}$. We also have
\begin{align*}
dA(\lambda) = \vert \varphi_z^\prime(w) \vert^2 dA(w)
\end{align*} so that we obtain
\begin{align*}
\int_\mathbb{D} \vert f(\varphi_z(w)) \vert^2 \vert \varphi_z^\prime(w) \vert^2 dA(w) = \int_\mathbb{D} \vert f(\lambda) \vert^2 dA(\lambda) = \int_D f(\lambda) \cdot \overline{f(\lambda)} dA(\lambda) = \langle f, f \rangle
\end{align*} Thus, we deduce that $U_z$ is a unitary operator (so $U_z^* = U_z^{-1}$). Next, they claim that $U_z^{-1} = U_z$. We have
\begin{align*}
U_z(U_z(f)) = U_z((f\circ \varphi_z)\varphi_z^\prime) = (((f \circ \varphi_z)\varphi_z^\prime)\circ \varphi_z) \varphi_z^\prime
\end{align*} Now, we compute
\begin{align*}
(((f\circ \varphi_z)\varphi_z^\prime) \circ \varphi_z (w)) \varphi_z^\prime(w) = (f(\varphi_z(\varphi_z(w))) \varphi_z^\prime(\varphi_z(w)))(\varphi_z^\prime(w)) = f(w) \cdot \frac{1}{\varphi_z^\prime(w)} \cdot \varphi_z^\prime(w) = f(w)
\end{align*} Thus we deduce that $U_zU_zf = f$ for all $f$ so that $U_z^* = U_z^{-1} = U_z$ and $U_z$ is actually self-adjoint. The authors define the Berezin transform of a symbol $u$ to be the Berezin transform of its Toeplitz operator $T_u$; that is, we have $\tilde{u} = \widetilde{T_u}$. For any orthogonal projection $P$, we know that $P = P^*$. Thus, we deduce that 
\begin{align*}
\tilde{u}(z) = \widetilde{T_u}(z) = \langle T_u k_z, k_z \rangle = \langle P(uk_z), k_z \rangle =\langle uk_z, P^*(k_z) \rangle = \langle uk_z, P(k_z) \rangle = \langle uk_z, k_z \rangle
\end{align*} It can be shown that
\begin{align*}
K_z(\lambda) = \frac{1}{(1 - \overline{z}\lambda)^2}
\end{align*} Furthermore, we note that
\begin{align*}
\Vert K_z \Vert_2^2 = \langle K_z, K_z \rangle = K_z(z) = \frac{1}{(1-\vert z \vert^2)^2}
\end{align*} so that
\begin{align*}
\Vert K_z \Vert_2 = \frac{1}{1- \vert z \vert^2}
\end{align*} Thus, we have
\[
k_z = \frac{K_z}{\Vert K_z \Vert_2} = (1-\vert z \vert^2)K_z
\] or
\[
k_z(\lambda) = (1- \vert z \vert^2)K_z(\lambda) = \frac{1-\vert z \vert^2}{(1-\overline{z}\lambda)^2}
\] Next, they claim that
\[
\vert \varphi_z^\prime(\lambda) \vert = \vert k_z(\lambda) \vert
\] To show this, we note that
\[
\varphi_z^\prime(\lambda) = \frac{-(1-\overline{z}\lambda) - (z-\lambda)(-\overline{z})}{(1-\overline{z}\lambda)^2} = \frac{\vert z \vert^2 - 1}{(1-\overline{z}\lambda)^2}
\] so that
\[
\vert \varphi_z^\prime(\lambda) \vert = \bigg\vert \frac{\vert z \vert^2 - 1}{(1-\overline{z}\lambda)^2} \bigg \vert = \frac{1-\vert z \vert^2}{\vert 1 - \overline{z}\lambda \vert^2} = \vert k_z(\lambda) \vert  
\] Now we note that
\[
\tilde{u}(z) = \langle uk_z, k_z \rangle = \int_\mathbb{D} u(w) \vert k_z(w) \vert^2 dA(w)
\] In the paper, they make the change of variables $\lambda = \varphi_z(w)$. From this, we obtain $dA(\lambda) = \vert \varphi_z^\prime(w) \vert^2 dA(w) = \vert k_z(w) \vert^2 dA(w)$ so that
\[
dA(w) = \frac{dA(\lambda)}{\vert k_z(w) \vert^2}
\] Since $\varphi_z \circ \varphi_z$ is the identity, we have $w = \varphi_z(\lambda)$. Thus we find that
\[
\tilde{u}(z) = \int_\mathbb{D} u(w) \vert k_z(w) \vert^2 dA(w) = \int_D u(\varphi_z(\lambda)) dA(\lambda)
\] The paper then discusses how the theorem can be used to prove several previously known results. First, we define Hankel operators. The Hankel operator $H_u: (L_a^2)^\perp \rightarrow L_a^2$ is defined as follows: 
\[
H_u f = (I - P)(uf)
\] The authors state that the main result of the paper can be used to show that $H_u$ is compact iff $\Vert H_u k_z \Vert_2 \rightarrow 0$ as $z \rightarrow \partial{\mathbb{D}}$ iff $\Vert u \circ \varphi_z - P(u \circ \varphi_z) \Vert_2 \rightarrow 0$ as $z \rightarrow \partial{\mathbb{D}}$. Before proving this, we show that
\[
H_{\overline{f}}^* H_g = T_{fg} - T_f T_g
\] First, we compute what $H_{\overline{f}}^*$ is. Let $a \in (L_a^2)^\perp$ and $b \in L_a^2$. Then, we have
\begin{align*}
&\langle H_{\overline{f}}^*a, b \rangle = \langle a, H_{\overline{f}} b \rangle = \langle a, (I-P)(\overline{f}b) \rangle = \langle a, \overline{f}b - P(\overline{f}b) \rangle = \langle a, \overline{f}b \rangle - \langle a,  P(\overline{f}b)\rangle \\
&= \langle fa, b \rangle - \langle Pa, \overline{f}b \rangle = \langle fa, b \rangle
\end{align*} Thus we deduce that $H_{\overline{f}}^*a = fa$. Now, we let $b \in L_a^2$. We obtain
\begin{align*}
&H_{\overline{f}}^* H_g(b) = P(H_{\overline{f}}^* H_g(b)) = P(H_{\overline{f}}^*(gb - P(gb))) = P(fgb - fP(gb)) = P(fgb) - P(fP(gb)) \\
&= T_{fg}(b) - T_f T_g(b)
\end{align*} thus proving that 
\[
H_{\overline{f}}^* H_g = T_{fg} - T_f T_g
\] so that we obtain
\[
H_u^* H_u = H_{\overline{\overline{u}}}^* H_u = T_{\overline{u}u} - T_{\overline{u}} T_u = T_{\vert u \vert^2} - T_{\overline{u}}T_u
\] Then, the authors state that $H_u$ being compact is equivalent to $H_u^* H_u$ being compact and that this occurs if and only if 
\[
\widetilde{H_u^* H_u}(z) \rightarrow 0 
\] as $z \rightarrow \partial{\mathbb{D}}$ (by the main theorem they will prove). Notice that
\[
\widetilde{H_u^* H_u}(z) = \langle H_u^* H_u k_z, k_z \rangle = \langle H_u k_z, H_u k_z \rangle = \Vert H_u k_z \Vert_2^2
\] so $H_u$ is compact if and only if $\Vert H_u k_z \Vert_2 \rightarrow 0$ as $z \rightarrow \partial{D}$. Next, we note that
\[
S_z = U_z S U_z = U_z(H_u^* H_u)U_z = U_z(T_{\vert u \vert^2} - T_{\overline{u}}T_u)U_z = U_z T_{\vert u \vert^2} U_z - U_z T_{\overline{u}}U_z U_zT_u U_z
\]
\[
= T_{\vert u \vert^2 \circ \varphi_z} - T_{\overline{u}\circ \varphi_z} T_{u \circ \varphi_z} = T_{\vert u \circ \varphi_z \vert^2} - T_{\overline{u \circ \varphi_z}} T_{u \circ \varphi_z} = H_{u \circ \varphi_z}^* H_{u \circ \varphi_z}
\] Thus we find that $H_u$ is compact iff $\Vert H_{u \circ \varphi_z}^* H_{u \circ \varphi_z} 1 \Vert_2 \rightarrow 0$ as $z \rightarrow \partial{\mathbb{D}}$. Now, we should note that
\begin{align*}
&\Vert H_{u \circ \varphi_z} 1 \Vert_2^2 = \langle H_{u \circ \varphi_z} 1, H_{u \circ \varphi_z} 1 \rangle = \langle H_{u \circ \varphi_z}^* H_{u \circ \varphi_z}  1, 1 \rangle \leq \Vert H_{u \circ \varphi_z}^* H_{u \circ \varphi_z} 1 \Vert_2 \Vert 1 \Vert_2 =\\
&\Vert H_{u \circ \varphi_z}^* H_{u \circ \varphi_z} 1 \Vert_2 \leq \Vert H_{u \circ \varphi_z}^* \Vert \Vert H_{u \circ \varphi_z} 1 \Vert_2 
\end{align*} This inequality shows that if 
\[
\Vert H_{u \circ \varphi_z}^* H_{u \circ \varphi_z} 1 \Vert_2 
\] goes to $0$, then
\[
\Vert H_{u \circ \varphi_z} 1 \Vert_2
\] also goes to $0$. It also shows that if 
\[
\Vert H_{u \circ \varphi_z} 1 \Vert_2
\] goes to $0$, then
\[
\Vert H_{u \circ \varphi_z}^* H_{u \circ \varphi_z} 1 \Vert_2
\] goes to $0$. However, we note that 
\[
H_{u \circ \varphi_z} 1 = (I-P)(u\circ \varphi_z 1) = u \circ \varphi_z - P(u \circ \varphi_z)
\] Thus we find that $H_u$ is compact if and only if $\Vert u \circ \varphi_z - P(u \circ \varphi_z) \Vert_2 \rightarrow 0$ as $z\rightarrow \partial{\mathbb{D}}$. 

\par Next, we note a lemma that they state in the paper, which says that
\[
\tilde{S} \circ \varphi_z = \widetilde{S_z}
\]
They explicitly leave the proof of this lemma as an exercise for the reader. First, I note that 
\[
\tilde{S} \circ \varphi_z (w) = \tilde{S}(\varphi_z(w)) = \langle Sk_{\varphi_z(w)}, k_{\varphi_z(w)} \rangle
\] and that 
\[
\widetilde{S_z}(w) = \langle S_z k_w, k_w \rangle = \langle U_z S U_z k_w, k_w \rangle = \langle S U_z k_w,  U_z k_w \rangle
\] (here I use the fact that $U_z$ is self-adjoint). We note that
\[
U_z k_w = (k_w \circ \varphi_z) \varphi_z^\prime
\]
Now, I appeal to the formula from the book Several Tasty Bits of Complex Variables: 
\[
K_U(z, \overline{\zeta}) = \det Df(z) \overline{\det Df(\zeta)}  K_V(f(z), \overline{f(\zeta)})
\]
 Here I take $U = V = \mathbb{D}$, $f = \varphi_z$, $z = w$ and $\zeta = \varphi_z(\lambda)$. Thus the above formula becomes 
 \[
 K_\mathbb{D}(w,\overline{\varphi_z(\lambda)}) = \det D \varphi_z(w) \overline{\det D \varphi_z(\varphi_z(\lambda))} K_\mathbb{D}(\varphi_z(w), \overline{\lambda})
 \] I compute 
 \[
 K_\mathbb{D}(w,\overline{\varphi_z(\lambda)}) = \overline{K_w(\varphi_z(\lambda))} = \overline{\frac{k_w(\varphi_z(\lambda))}{1 - \vert w \vert^2}}
 \] 
 \[
 K_\mathbb{D}(\varphi_z(w), \overline{\lambda}) = \overline{K_{\varphi_z(w)}(\lambda)} = \overline{\frac{k_{\varphi_z(w)}(\lambda)}{1- \vert \varphi_z(w) \vert^2}}
 \] 
 \[
 \det D \varphi_z(w) = \varphi_z^\prime(w) = \frac{\vert z \vert^2 - 1}{(1-\overline{z}w)^2}
 \] and finally 
 \[
 \overline{\det D \varphi_z(\varphi_z(\lambda))} = \overline{\varphi_z^\prime(\varphi_z(\lambda))} = \overline{\frac{1}{\varphi_z^\prime(\lambda)}}
 \] Putting it all together, I obtain 
 \[\overline{\frac{k_w(\varphi_z(\lambda))}{1 - \vert w \vert^2}} =  \frac{\vert z \vert^2 - 1}{(1-\overline{z}w)^2} \cdot  \overline{\frac{1}{\varphi_z^\prime(\lambda)}} \cdot  \overline{\frac{k_{\varphi_z(w)}(\lambda)}{1- \vert \varphi_z(w) \vert^2}}
 \] I take the complex conjugate to obtain 
 \[
 \frac{k_w(\varphi_z(\lambda))}{1 - \vert w \vert^2} =  \frac{\vert z \vert^2 - 1}{(1-z\overline{w})^2} \cdot  \frac{1}{\varphi_z^\prime(\lambda)} \cdot  \frac{k_{\varphi_z(w)}(\lambda)}{1- \vert \varphi_z(w) \vert^2}
\] Using the fact that 
\[
1 - \vert \varphi_z(w)\vert^2 = \frac{(1-\vert z \vert^2)(1- \vert w \vert^2)}{\vert 1 - z \overline{w} \vert^2}
\] the above equation becomes 
\[
\frac{k_w(\varphi_z(\lambda))}{1 - \vert w \vert^2} =  \frac{\vert z \vert^2 - 1}{(1-z\overline{w})^2} \cdot  \frac{1}{\varphi_z^\prime(\lambda)} \cdot  \frac{k_{\varphi_z(w)}(\lambda)}{\frac{(1-\vert z \vert^2)(1- \vert w \vert^2)}{\vert 1 - z \overline{w} \vert^2}}
\] Simplification yields 
\[
k_w(\varphi_z(\lambda)) \varphi_z^\prime(\lambda) = k_{\varphi_z(w)}(\lambda) \cdot -\frac{\vert 1- z \overline{w} \vert^2}{(1 - z \overline{w})^2}
\] Now, we note that
\begin{align*}
&\widetilde{S_z}(w) = \langle SU_z k_w,  U_z k_w\rangle = \int_\mathbb{D} S(U_z k_w(\lambda)) \overline{U_z k_w(\lambda)} dA(\lambda) \\
&= \int_\mathbb{D} S(k_w(\varphi_z(\lambda)) \varphi_z^\prime(\lambda)) \cdot \overline{k_w(\varphi_z(\lambda)) \varphi_z^\prime(\lambda)} dA(\lambda) \\
&=  \int_\mathbb{D} Sk_{\varphi_z(w)}(\lambda) \cdot -\frac{\vert 1- z \overline{w} \vert^2}{(1 - z \overline{w})^2} \cdot \overline{k_{\varphi_z(w)}(\lambda)} \cdot  \overline{-\frac{\vert 1- z \overline{w} \vert^2}{(1 - z \overline{w})^2}} dA(\lambda) \\
&= \int_\mathbb{D} Sk_{\varphi_z(w)}(\lambda) \overline{k_{\varphi_z(w)}(\lambda)} dA(\lambda)\\  
&= \langle Sk_{\varphi_z(w)}, k_{\varphi_z(w)} \rangle = \tilde{S} \circ \varphi_z (w) 
\end{align*}
\section*{My Work}
On standard spaces, it has been shown that if the symbol $\psi$ vanishes on the boundary of the unit disk $\mathbb{D}$, then the operator $T_\psi$ is compact. I wanted to determine whether or not this holds true on exponentially weighted spaces. In particular, I let $\lambda(z) = \exp(-1/(1-\vert z \vert^2))$, and I considered the Bergman space $A^2(\Omega,\lambda)$. Notice that $\lambda(z)$ tends to $0$ very quickly as $z \rightarrow \partial{\mathbb{D}}$. Thus, there are many functions that are square-integrable with respect to this weight. In a sense, the weighted Bergman space is much larger than the unweighted Bergman space. Thus, it might be possible to find a symbol $\phi$ that vanishes on $\partial{\mathbb{D}}$ such that $T_\phi$ is not compact. This was my goal throughout the project. Now, one characterization of compactness is that an operator $S$ is compact if and only if it takes bounded sequences to a sequence with a converging subsequence. Thus, we will be attempting to find a sequence of bounded functions $\{f_n\}$ such that $\{T_\phi(f_n)\}$ does not contain a converging subsequence. 
Let the symbol $\phi$ be defined by $\phi(z) = 1 - \vert z \vert^2$. 
\par I first consider the sequence $\{f_n\}_{n \in \N}$ where $f_n(z) = z^n$. I note that
\begin{align*}
&T_\phi(f_m)(z) =  P(\phi f_m)(z) = \int_\mathbb{D} B_\lambda(z,w) f_m(w) \phi(w) d\lambda(w) = \int_\mathbb{D} \sum_{n=0}^\infty \frac{z^n \overline{w}^n}{\Vert z^n \Vert^2} w^m (1 - \vert w \vert^2) d\lambda(w) \\
&= \frac{z^m}{\Vert z^m \Vert^2} \int_D (\vert w \vert^{2m} - \vert w \vert^{2m+2}) d\lambda(w) = \frac{z^m}{\Vert z^m \Vert^2} (\Vert z^m \Vert^2 - \Vert z^{m+1} \Vert^2) = z^m \bigg( 1 - \frac{\Vert z^{m+1}\Vert^2}{\Vert z^m \Vert^2}\bigg)
\end{align*} A couple things to note here. First, $B_\lambda(z,w)$ represents the Bergman kernel. We can expand it into a series by Prop 5.2.5 from Tasty Bits of Several Complex Variables. Also, we do implicitly interchange integration and summation in the above equation. Finally, the norm we are using is the norm of the exponentially weighted Bergman space. Now, it is clear that  $\Vert T_\phi(f_m) \Vert \leq \Vert f_m \Vert$ for all $m \in \N$; that is the sequence $\{T_\phi(f_n)\}_{n \in \N}$ converges to the function $0$.
\par This sequence has not given desirable results, but I believe this is because I have not been using the exponential weight to my advantage. Now, I begin to try functions with singularities on the boundary. First, I try the sequence $\{f_n\}$ where
\[
f_n(z) = \frac{1}{(1-z)^n}
\] so that $\{f_n\}$ is a sequence of functions that are analytic in the unit disk. We may try to compute the weighted norm of these functions as follows:
\begin{align*}
&\Vert f_1 \Vert^2 = \int_D \frac{1}{\vert 1 - z \vert^2} \lambda(z) dA(z) = \frac{1}{\pi} \int_0^{2\pi} \int_0^1 \frac{r}{\vert 1 - re^{i\theta} \vert^2} \exp\bigg(-\frac{1}{1-r^2}\bigg) dr d\theta \approx 0.219384 \\
&\Vert f_2 \Vert^2 = \int_D \frac{1}{\vert 1 - z \vert^4} \lambda(z) dA(z) = \frac{1}{\pi}\int_0^{2\pi} \int_0^1 \frac{r}{\vert 1 - re^{i\theta} \vert^4} \exp\bigg(-\frac{1}{1-r^2}\bigg) dr d\theta \approx 1.10366 \\
&\Vert f_3 \Vert^2 = \int_D \frac{1}{\vert 1 - z \vert^6} \lambda(z) dA(z) = \frac{1}{\pi}\int_0^{2\pi} \int_0^1 \frac{r}{\vert 1 - re^{i\theta} \vert^6} \exp\bigg(-\frac{1}{1-r^2}\bigg) dr d\theta \approx 25.0158 \\
&\Vert f_4 \Vert^2 = \int_D \frac{1}{\vert 1 - z \vert^{8}} \lambda(z) dA(z) = \frac{1}{\pi}\int_0^{2\pi} \int_0^1 \frac{r}{\vert 1 - re^{i\theta} \vert^{8}} \exp\bigg(-\frac{1}{1-r^2}\bigg) dr d\theta \approx 1700 \\
&\Vert f_5 \Vert^2 = \int_D \frac{1}{\vert 1 - z \vert^{10}} \lambda(z) dA(z) = \frac{1}{\pi} \int_0^{2\pi} \int_0^1 \frac{r}{\vert 1 - re^{i\theta} \vert^{10}} \exp\bigg(-\frac{1}{1-r^2}\bigg) dr d\theta \approx 262325
\end{align*} Here we can see that the sequence of norms is again increasing and that it appears to be unbounded. Thus, we don't need to consider the images of the functions. 
\par I will now try the following sequence: $f_n(z) = 1/(1-z^n)$. First, we must ensure that $\{f_n\}$ is bounded. We compute
\begin{align*}
&\Vert f_1 \Vert^2 = \int_D \frac{1}{\vert 1 - z \vert^2} \lambda(z) dA(z) = \frac{1}{\pi} \int_0^{2\pi} \int_0^1 \frac{r}{\vert 1 - re^{i\theta} \vert^2} \exp\bigg(-\frac{1}{1-r^2}\bigg) dr d\theta \approx 0.219384 \\
&\Vert f_2 \Vert^2 = \int_D \frac{1}{\vert 1 - z^2 \vert^2} \lambda(z) dA(z) = \frac{1}{\pi} \int_0^{2\pi} \int_0^1 \frac{r}{\vert 1 - (re^{i\theta})^2 \vert^2} \exp\bigg(-\frac{1}{1-r^2}\bigg) dr d\theta \approx 0.169760 \\
&\Vert f_3 \Vert^2 = \int_D \frac{1}{\vert 1 - z^3 \vert^2} \lambda(z) dA(z) = \frac{1}{\pi} \int_0^{2\pi} \int_0^1 \frac{r}{\vert 1 - (re^{i\theta})^3 \vert^2} \exp\bigg(-\frac{1}{1-r^2}\bigg) dr d\theta \approx 0.157491\\
&\Vert f_4 \Vert^2 = \int_D \frac{1}{\vert 1 - z^4 \vert^2} \lambda(z) dA(z) = \frac{1}{\pi} \int_0^{2\pi} \int_0^1 \frac{r}{\vert 1 - (re^{i\theta})^4 \vert^2} \exp\bigg(-\frac{1}{1-r^2}\bigg) dr d\theta \approx 0.152963\\
&\Vert f_5 \Vert^2 = \int_D \frac{1}{\vert 1 - z^5 \vert^2} \lambda(z) dA(z) = \frac{1}{\pi} \int_0^{2\pi} \int_0^1 \frac{r}{\vert 1 - (re^{i\theta})^5 \vert^2} \exp\bigg(-\frac{1}{1-r^2}\bigg) dr d\theta \approx 0.150943
\end{align*}
The sequence $\{f_n\}$ appears to be bounded, so  we may try to compute $T_\phi(f_n)$ for several values of $n$. Let $n = 1$. Then
\begin{align}
T_\phi(f_1)(z) = P(\phi f_1)(z) = \int_\mathbb{D} B_\lambda(z,w) f_1(w) \phi(w) d\lambda(w)
\end{align}
We note that
\begin{align}
&f_1(w) \phi(w) = \frac{1}{1-w} \cdot (1-w\overline{w}) = (1 + w + w^2 + w^3 + \cdots)(1 - w\overline{w}) 
\\ &= 1 + w + w^2 + w^3 + \cdots - w \overline{w} - w^2 \overline{w} - w^3 \overline{w} - w^4 \overline{w} - \cdots
\end{align} and that

\begin{align}
B_\lambda(z,w) = \frac{z^0 \overline{w}^0}{\Vert z^0 \Vert^2} + \frac{z^1 \overline{w}^1}{\Vert z^1 \Vert^2} + \frac{z^2 \overline{w}^2}{\Vert z^2 \Vert^2} + \cdots 
\end{align}

 Now, we have

\begin{align}
&\frac{z^0}{\Vert z^0 \Vert^2} \overline{w}^0 (1 + w + w^2 + w^3 + \cdots - w\overline{w} - w^2\overline{w} - w^3 \overline{w} - w^4 \overline{w} - \cdots )=\\
& \frac{1}{\Vert 1 \Vert^2} (1 + w + w^2 + w^3 + \cdots - w\overline{w} - w^2\overline{w} - w^3 \overline{w} - w^4 \overline{w} -\cdots)
\end{align}

\begin{align}
&\frac{z^1}{\Vert z^1 \Vert^2} \overline{w}^1 (1 + w + w^2 + w^3 + \cdots - w\overline{w} - w^2\overline{w} - w^3 \overline{w} - w^4 \overline{w} - \cdots )=\\
& \frac{z}{\Vert z \Vert^2} (\overline{w} + w\overline{w} + w^2\overline{w} + w^3\overline{w} + \cdots - w\overline{w}^2 - w^2\overline{w}^2 - w^3 \overline{w}^2 - w^4 \overline{w}^2 -\cdots)
\end{align}

\begin{align}
&\frac{z^2}{\Vert z^2 \Vert^2} \overline{w}^2 (1 + w + w^2 + w^3 + \cdots - w\overline{w} - w^2\overline{w} - w^3 \overline{w} - w^4 \overline{w} - \cdots )=\\
& \frac{z^2}{\Vert z^2 \Vert^2} (\overline{w}^2 + w\overline{w}^2 + w^2\overline{w}^2 + w^3\overline{w}^2 + \cdots - w\overline{w}^3 - w^2\overline{w}^3 - w^3 \overline{w}^3 - w^4 \overline{w}^3 -\cdots)
\end{align}

\begin{align}
&\frac{z^3}{\Vert z^3 \Vert^2} \overline{w}^3 (1 + w + w^2 + w^3 + \cdots - w\overline{w} - w^2 \overline{w} - w^3 \overline{w} - w^4 \overline{w} - \cdots) = \\
& \frac{z^3}{\Vert z^3 \Vert^2} (\overline{w}^3 + w\overline{w}^3 + w^2\overline{w}^3 + w^3\overline{w}^3 + \cdots - w\overline{w}^4 - w^2\overline{w}^4 - w^3 \overline{w}^4 - w^4 \overline{w}^4 -\cdots)
\end{align}
If we assume that we can interchange summation and integration, then every term with $w^a\overline{w}^b$ vanishes for $a \neq b$, so we obtain
\begin{align}
&\frac{1}{\Vert 1 \Vert^2} \int_\mathbb{D} (1 - w\overline{w}) d\lambda(w) + \frac{z}{\Vert z \Vert^2} \int_\mathbb{D} (w \overline{w} - w^2 \overline{w}^2) d\lambda(w) + \frac{z^2}{\Vert z^2 \Vert^2} \int_\mathbb{D} (w^2\overline{w}^2 - w^3 \overline{w}^3) d\lambda(w) + \cdots = \\
& \frac{1}{\Vert 1 \Vert^2} \int_\mathbb{D} (1 - \vert w \vert^2) d\lambda(w) + \frac{z}{\Vert z \Vert^2} \int_\mathbb{D} (\vert w \vert^2 - \vert w \vert^4) d\lambda(w) + \frac{z^2}{\Vert z^2 \Vert^2} \int_\mathbb{D} (\vert w \vert^4 - \vert w \vert^6) d\lambda(w) + \cdots = \\
& \frac{1}{\Vert 1 \Vert^2}(\Vert 1 \Vert^2 - \Vert z \Vert^2) + \frac{z}{\Vert z \Vert^2}(\Vert z \Vert^2 - \Vert z^2 \Vert^2) + \frac{z^2}{\Vert z^2 \Vert^2}(\Vert z^2 \Vert^2 - \Vert z^3 \Vert^2) + \cdots = \\
& \bigg(1 - \frac{\Vert z \Vert^2}{\Vert 1 \Vert^2}\bigg) + z\bigg(1 - \frac{\Vert z^2 \Vert^2}{\Vert z \Vert^2}\bigg) + z^2 \bigg(1 - \frac{\Vert z^3 \Vert^2}{{\Vert z^2 \Vert^2}}\bigg) + \cdots = \\
& \sum_{i=0}^\infty z^i\bigg(1 - \frac{\Vert z^{i+1} \Vert^2}{\Vert z^i \Vert^2}\bigg)
\end{align} Now, we let $n = 2$. We have
\begin{align}
T_\phi(f_2)(z) = P(\phi f_2)(z) = \int_\mathbb{D} B_\lambda(z,w) f_2(w) \phi(w) d\lambda(w)
\end{align} We compute 
\begin{align}
&f_2(w) \phi(w) = \frac{1 - w\overline{w}}{1-w^2} = (1 + w^2 + w^4 + w^6 + \cdots)(1 - w \overline{w}) \\
& = 1 + w^2 + w^4  + w^6 + \cdots - w\overline{w} - w^3 \overline{w} - w^5 \overline{w} - w^7 \overline{w} - \cdots
\end{align} We recall that
\begin{align}
B_\lambda(z,w) = \sum_{j=0}^\infty \frac{z^j \overline{w}^j}{\Vert z^j \Vert^2}
\end{align} Now we compute
\begin{align}
&\frac{z^0}{\Vert z^0 \Vert^2}\overline{w}^0(1 + w^2 + w^4  + w^6 + \cdots - w\overline{w} - w^3 \overline{w} - w^5 \overline{w} - w^7 \overline{w} - \cdots) = \\
& \frac{1}{\Vert 1 \Vert^2} (1 + w^2 + w^4 + w^6 + \cdots - w\overline{w} - w^3\overline{w} - w^5 \overline{w} - w^7 \overline{w} - \cdots)
\end{align}
\begin{align}
&\frac{z^1}{\Vert z^1 \Vert^2} \overline{w}^1 (1 + w^2 + w^4  + w^6 + \cdots - w\overline{w} - w^3 \overline{w} - w^5 \overline{w} - w^7 \overline{w} - \cdots) = \\ 
& \frac{z}{\Vert z \Vert^2} (\overline{w} + w^2 \overline{w} + w^4\overline{w} + w^6 \overline{w} + \cdots - w \overline{w}^2 - w^3 \overline{w}^2 - w^5 \overline{w}^2 - w^7 \overline{w}^2 - \cdots)
\end{align}
\begin{align}
&\frac{z^2}{\Vert z^2 \Vert^2} \overline{w}^2 (1 + w^2 + w^4 + w^6 + \cdots - w\overline{w} - w^3 \overline{w} - w^5 \overline{w} - w^7 \overline{w} - \cdots) \\
& = \frac{z^2}{\Vert z^2 \Vert^2}  (\overline{w}^2 + w^2\overline{w}^2 + w^4 \overline{w}^2 + w^6 \overline{w}^2 + \cdots - w \overline{w}^3 - w^3 \overline{w}^3 - w^5 \overline{w}^3 - w^7\overline{w}^3 - \cdots)
\end{align} Assuming this pattern continues and interchanging integration and summation, we obtain
\begin{align}
&\frac{1}{\Vert 1 \Vert^2} \int_\mathbb{D} (1 - \vert w \vert^2) d\lambda(w) + \frac{z^2}{\Vert z^2 \Vert^2} \int_\mathbb{D} (\vert w \vert^4 - \vert w \vert^6) d\lambda(w) + \cdots  = \\
& \frac{1}{\Vert 1 \Vert^2}(\Vert 1 \Vert^2 - \Vert z \Vert^2) + \frac{z^2}{\Vert z^2 \Vert^2}(\Vert z^2 \Vert^2 - \Vert z^3\Vert^2) + \cdots =\\
& \sum_{i=0}^\infty z^{2i} \bigg(1 - \frac{\Vert z^{2i+1}\Vert^2}{\Vert z^{2i}\Vert^2}\bigg)
\end{align} Now let $n = 3$. Note that
\begin{align}
&f_3(w) = \frac{1}{1-w^3} = 1 + w^3 + w^6 + w^9 + \cdots\\
&\phi(w) = 1 - w \overline{w}\\
&B_\lambda(z,w) = \frac{z^0 \overline{w}^0}{\Vert z^0 \Vert^2} + \frac{z^1 \overline{w}^1}{\Vert z^1 \Vert^2} + \frac{z^2 \overline{w}^2}{\Vert z^2 \Vert^2} + \cdots
\end{align} We have
\begin{align}
&f_3(w) \phi(w) = \frac{1-w\overline{w}}{1-w^3} = (1+w^3 + w^6 + w^9 + \cdots)(1 - w\overline{w}) = \\
& 1 + w^3 + w^6 + w^9 + \cdots - w\overline{w} - w^4 \overline{w} - w^7 \overline{w} - w^{10} \overline{w} - \cdots
\end{align} Note that
\begin{align}
&\frac{z^0}{\Vert z^0 \Vert^2} \overline{w}^0 (1 + w^3 + w^6 + w^9 + \cdots - w\overline{w} - w^4 \overline{w} - w^7 \overline{w} - w^{10} \overline{w} - \cdots) = \\
&\frac{1}{\Vert 1 \Vert^2}(1 + w^3 + w^6 + w^9 + \cdots - w\overline{w} - w^4 \overline{w} - w^7 \overline{w} - w^{10} \overline{w} - \cdots)
\end{align}
\begin{align}
&\frac{z^1}{\Vert z^1 \Vert^2} \overline{w}^1 (1 + w^3 + w^6 + w^9 + \cdots - w\overline{w} - w^4 \overline{w} - w^7 \overline{w} - w^{10} \overline{w} - \cdots) = \\
&\frac{z}{\Vert z \Vert^2}(\overline{w} + w^3\overline{w} + w^6\overline{w} + w^9\overline{w} + \cdots - w\overline{w}^2 - w^4 \overline{w}^2 - w^7 \overline{w}^2 - w^{10} \overline{w}^2 - \cdots)
\end{align}
\begin{align}
&\frac{z^2}{\Vert z^2 \Vert^2} \overline{w}^2 (1 + w^3 + w^6 + w^9 + \cdots - w\overline{w} - w^4 \overline{w} - w^7 \overline{w} - w^{10} \overline{w} - \cdots) = \\
&\frac{z^2}{\Vert z^2 \Vert^2}(\overline{w}^2 + w^3\overline{w}^2 + w^6\overline{w}^2 + w^9\overline{w}^2 + \cdots - w\overline{w}^3 - w^4 \overline{w}^3 - w^7 \overline{w}^3 - w^{10} \overline{w}^3 - \cdots)
\end{align}
\begin{align}
&\frac{z^3}{\Vert z^3 \Vert^2} \overline{w}^3 (1 + w^3 + w^6 + w^9 + \cdots - w\overline{w} - w^4 \overline{w} - w^7 \overline{w} - w^{10} \overline{w} - \cdots) = \\
&\frac{z^3}{\Vert z^3 \Vert^2}(\overline{w}^3 + w^3\overline{w}^3 + w^6\overline{w}^3 + w^9\overline{w}^3 + \cdots - w\overline{w}^4 - w^4 \overline{w}^4 - w^7 \overline{w}^4 - w^{10} \overline{w}^4 - \cdots)
\end{align} Assuming the pattern continues and interchanging the integration and summation, we obtain
\begin{align}
&\frac{1}{\Vert 1 \Vert^2} \int_\mathbb{D} (1 - \vert w \vert^2) d\lambda(w) + \frac{z^3}{\Vert z^3 \Vert^2}  \int_\mathbb{D} (\vert w \vert^6 - \vert w \vert^8) d\lambda(w) + \cdots = \\
&\frac{1}{\Vert 1 \Vert^2} (\Vert 1 \Vert^2 - \Vert z \Vert^2) + \frac{z^3}{\Vert z^3 \Vert^2} (\Vert z^3 \Vert^2 - \Vert z^4 \Vert^2) + \cdots = \\
& \sum_{i=0}^\infty z^{3i}\bigg(1 - \frac{\Vert z^{3i+1}\Vert^2}{\Vert z^{3i}\Vert^2}\bigg)
\end{align} Similarly, we find that for any $n \in \mathbb{N}$, we have
\begin{align}
T_\phi(f_n)(z) = \sum_{i=0}^\infty z^{ni} \bigg(1 - \frac{\Vert z^{ni+1}\Vert^2}{\Vert z^{ni}\Vert^2}\bigg) 
\end{align}
\end{document} 